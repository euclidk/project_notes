\documentclass[a4paper]{article}
\usepackage{amsmath}
\usepackage{parskip}

\author{Euclid Kosasih}
\date{September 1, 2017}
\title{Notes for ELEC4402 Project}

\begin{document}
\maketitle
\section{Noise PSD}
\begin{equation}
	N_0 = \frac{(V_{av})^2}{R}
\end{equation}
Where $N_0$ is the power spectral density (PSD) of the noise signal in $W/Hz$ and $V_{av}$ is the average voltage of the \textit{noise} signal in the frequency domain (use FFT). Note that $V_{av}$ is measured using the \texttt{Wideband True RMS module}, it is not calculated.

\section{Clock Signal}
Display the clock signal to measure $R_b$. Do not rely on ``eyeballing'' 1 bit on the modulated signal. The clock signal may be weak, so if this is the case, use the \texttt{ADDER} module to amplify the signal. 

The master clock is 8.333kHz, however, the bit clock is 2kHz, since
\begin{equation}
	R_b = \frac{MCLK}{4}
\end{equation}

\section{BER Calculations}
\begin{equation}
	BER = \frac{N_e}{N_t}
\end{equation}
Where $BER$ is the bit error rate, $N_e$ is the number of erroneous bits and $N_b$ is the total number of bits transmitted.

\begin{equation}
	\frac{E_b}{N_0} = \left(\frac{V_{av,s}}{V_{av,n,f}}\right)^2T_b
\end{equation}
Where $\frac{E_b}{N_0}$ is the signal to noise ratio (SNR), $V_{av,s}$ is the average voltage of the signal, $V_{av,n,f}$ is the average voltage of the noise signal in the frequency domain and $T_b$ is the time period for each bit.

Note that the graph used is log-normal. Example calculation:
\begin{align}
	BER &= \log_{10}{\left(\frac{538}{2000}\right)}\\
	\frac{E_n}{N_0}\ \mathrm{dB} &= 10\log_{10}
\end{align}

\section{Signal Constellation}
The signal constellation gives information on the symbol distance $d$ which in turn gives information on the bit error rate. 
	\begin{equation}
		BER = Q\left(\sqrt{\frac{d}{\cdots}}\right) \label{ber_d}
	\end{equation}
As $d$ decreases in eqn.~\eqref{ber_d}, $BER$ will increase.

\section{Troubleshooting}
\begin{itemize}
	\item Reset the \texttt{Line Decoder} each time a new clock is given.
	\item Use a local signal generator to avoid propagation delay.
	\item Use the \texttt{Phase Shifter} module $\phi = \pm 180^{\circ}$ if the signal is out of sync.
	\item Use the output of the \texttt{Error Counting Utilities} to reset the signal generator if message signal and demodulated signal are not in sync.
\end{itemize}

\section{Common Lab Test Questions}
\begin{enumerate}
	\item Sketch the (provided) modulated signal
	\item Demodulate the signal and sketch its spectra and temporal waveform
	\item Theory question on performance
	\item Identify which line encoding method was used
	\item Sketch block diagram of demodulator
\end{enumerate}

\section{Miscellaneous Notes for Future Labs}
\begin{itemize}
	\item Know which input/output of which module corresponds to the message, modulating, carrier and modulated signals.
	\item Know the demodulator circuit well since this will be the one used in the test (you will not be modulating your own signal).
\end{itemize}

\end{document}
